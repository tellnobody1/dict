\documentclass[twoside]{article}
\usepackage[paperwidth=13cm,paperheight=20cm,top=2cm,bottom=2cm,left=1.2cm,right=1.2cm,columnsep=2cm]{geometry}
\setcounter{secnumdepth}{0}
\setlength{\parindent}{3mm}

\usepackage{fancyhdr,extramarks,xstring}
\pagestyle{fancy}

\fancyhead[LE,LO]{\MakeUppercase{\firstxmark}}
\fancyhead[CE,CO]{\MakeUppercase{\lastxmark}}
\fancyhead[RE]{}
\fancyhead[RO]{\StrLeft{\lastxmark}{1}[\x]\textbf{\MakeUppercase{\x}}}
\renewcommand{\headrulewidth}{0pt}

\fancyfoot[L,C]{}
\fancyfoot[LE,RO]{\thepage}
\renewcommand{\footrulewidth}{0pt}

% fonts
\usepackage[english,ukrainian]{babel}
\usepackage{fontspec}
\setmainfont{Libertinus Serif}
\setsansfont{Libertinus Sans}
\setmonofont{Libertinus Mono}
\usepackage{unicode-math}
\setmathfont{Libertinus Math}

\newcommand{\entry}[2]{\extramarks{#1}{#1}\textbf{#1} \ #2.}
\newcommand{\subentry}[2]{\indent\textbf{#1} \ #2.}

\usepackage{hyperref}


\begin{document}
\title{\uppercase{Англо-\\український\\словник\\сучасних\\слів}}
\author{@tellnobody1}
\maketitle

\thispagestyle{empty}
\clearpage

\begin{center}
\section{\uppercase{Англійський алфавіт}}
\begin{tabular}{| c c c c c c c |}
  \hline
  Aa & Ee & Ii & Mm & Qq & Uu & Yy \\
  Bb & Ff & Jj & Nn & Rr & Vv & Zz \\
  Cc & Gg & Kk & Oo & Ss & Ww &    \\
  Dd & Hh & Ll & Pp & Tt & Xx &    \\
  \hline
\end{tabular}
\end{center}

\thispagestyle{empty}
\clearpage

\begin{twocolumn}

\section{@}
\entry{@}{равлик}

\section{A}
\entry{accessory}{причанда́л}\\
\entry{account}{обліковий запис}\\
\entry{accumulator}{накопичувач}\\
\entry{add-on}{додаток}\\
\hypertarget{animation}{\entry{animation}{(\textit{пр.}) ожи́\-ва, оживанка}}\\
\entry{anime}{японська анімація, аніме; \textit{див.} \hyperlink{animation}{animation}}\\
\entry{app}{\textit{див.} \hyperlink{app}{application}}\\
\hypertarget{app}{\entry{application}{застосунок, програма}}\\
\entry{assistant}{помі́чни́к, пома\-га́ч}

\section{B}
\entry{backpressure, back pressure}{проти\-ти́ск, зворо́тний тиск, контрти́ск}\\
\entry{barrier}{рога́тка, зворотниця, заста́ва, пере́пи́н, шлаг\-ба́\-ум}\\
\entry{benchmark}{контро́льне завда́ння́}\\
\entry{blackout}{за́тьма}\\
\entry{branch}{\textbf{1.} (\textit{розгалуження чогось}) ві́тка; \textbf{2.} (\textit{галузь}) па́рость (\textit{р.} -ти); \textbf{3.} (\textit{ботаніка}) ві́тка; (\textit{головна}) відно́га. \textit{Для мн.} ві́ти, \textit{збір.} ві́ття, (\textit{дрібні}) пагі́лля}\\
\entry{brown}{цинамо́новий}\\
\entry{browser}{перегля́дач}\\
\hypertarget{bug}{\entry{bug}{(\textit{у коді}) хиба, (\textit{у програмі}) вада, дефект}}\\
\entry{byte}{(\textit{беззна́кове ціле число від 0 до $2^8$-1 включно}) байт}

\section{C}
\entry{cafe}{кава́рня}\\
\entry{caffeine}{кавеї́н (-ну)}\\
\hypertarget{css}{\entry{Cascading Style Sheets}{стилі}}\\
\hypertarget{coiffure}{\entry{coiffure}{за́чіска}}\\
\entry{calculator}{рахівник}\\
\entry{change}{дрібні́ гро́ші (-шей і -шів), дрібняки́ (-кі́в), дрі́б’язок, (рідше) дрібно́та, дрібно́тка}\\
\entry{commit}{(\textit{пр.: процес}) вно́сити, (\textit{пр.: результат}) вне́сок}\\
\entry{compilation}{збірка, збирати}\\
\entry{compile}{збирати}\\
\entry{compiler}{збирач}\\
\entry{computer}{(\textit{пр.}) об\-чи́с\-лю\-вач}\\
\hypertarget{laptopcomputer}{\subentry{laptop \textasciitilde}{(\textit{пр.}) похватни́й обчи́слювач}}\\
\entry{coprime}{навзає́м про́сті}\\
\entry{CPU}{пристрій центрального обробля́ння, обробник}\\
\entry{creator}{творе́ць}\\
\entry{CSS}{\textit{див.} \hyperlink{css}{Cascading Style Sheets}}\\
\entry{cybernetics}{кіберне́тика}\\
\entry{cyberpunk}{(\textit{пр.}) кіберпа́нщина}

\section{D}
\entry{deadline}{речене́ць}\\
\entry{default}{типовий}\\
\subentry{by \textasciitilde}{типово}\\
\entry{definition}{визначення}\\
\entry{defect}{\textit{див.} \hyperlink{bug}{bug}}\\
\entry{delivery}{дові́з (\textit{р.} -во́зу)}\\
\entry{desktop}{стільни́ця; стільни́чний, насті́льний}\\
\entry{developer}{розро́бник}\\
\entry{device}{пристрі́й, при́лад}\\
\entry{directory}{до́ві́дни́к, \textit{див.} \hyperlink{folder}{folder}}\\
\entry{dislike}{(\textit{пр.}) невподоба}\\
\entry{download}{виванта́жування, \textit{док.} ви́вантаження; виванта́\-жувати, -ся, ви́вантажити,\\-ся}\\
\entry{driver}{(\textit{пр.}) керун}

\section{E}
\entry{easter egg}{(\textit{пр.}) яй\-це́-рай\-це́, (\textit{пр.}) прихованка}\\
\entry{electricity}{громови́на}\\
\entry{engine}{двигу́н, двига́ч, руші́й (\textit{р.} рушія́), руши́тель}\\
\subentry{search \textasciitilde}{(\textit{пр.}) руші́й по́шуку}\\
\entry{Europe}{Евро́па}

\section{F}
\entry{fantasy}{(\textit{пр.}) химе́рство}\\
\hypertarget{folder}{\entry{folder}{те́ка}}\\
\entry{feature}{засіб, функція}\\
\entry{FOSS}{\textit{див.} \hyperlink{foss}{Free and Open-Source Software}}\\
\entry{framework}{кістя́к}\\
\hypertarget{foss}{\entry{Free and Open-Source Software}{вільне та відкрите програмне забезпечення [ВВПЗ]}}

\section{G}
\entry{game}{гра}\\
\subentry{action role-playing \textasciitilde}{рольовий бойовик}\\
\subentry{artillery \textasciitilde}{артиле́рія}\\
\subentry{board \textasciitilde}{насті́льна гра}\\
\hypertarget{rpg}{\subentry{role-playing \textasciitilde}{рольова відеогра}}\\
\subentry{tactical role-playing \textasciitilde}{тактична рольова гра}\\
\subentry{video \textasciitilde}{відеогра}\\
\entry{graceful shutdown}{(\textit{пр.}) ґре́чна при́пинка}

\section{H}
\entry{haircut}{(\textit{процес}) стри́ження, (\textit{результат}) стрижба́}\\
\hypertarget{hairdo}{\entry{hairdo}{\textit{див.} \hyperlink{coiffure}{coiffure}}}\\
\entry{hairstyle}{\textit{див.} \hyperlink{hairdo}{hairdo}}\\
\entry{handler}{\textbf{1.} маніпуля́тор; при́ст\-рій керува́ння, керува́ч; \textbf{2.} опрацьо́вувач, програ́ма опра\-цьо́\-вування}\\
\hypertarget{eventhandler}{\subentry{event \textasciitilde}{програ́ма опра\-цьо́\-вування [опрацьо́вувач] поді́й}}\\
\entry{hardcover}{паляту́рка, паліту́рка}\\
\entry{hash}{(\textit{пр.}) криш}\\
\subentry{\textasciitilde\ table}{(\textit{пр.}) криштаблиця}\\
\entry{hashtag}{(\textit{пр.}) кришмітка}\\
\entry{help}{по́міч}\\
\entry{host}{(\textit{пр.}) відник}

\section{I}
\entry{implementation}{реаліза́ція}\\
\entry{influencer}{(\textit{пр.}) впливач (-ка), (\textit{пр.}) на\-мо́\-жник (-иця)}

\section{L}
\entry{laptop computer}{\textit{див.} \hyperlink{laptopcomputer}{computer}}\\
\entry{listener}{\textbf{1.} слухач; \textbf{2.} (\textit{в мовах Java, JavaScript}) див. \hyperlink{eventhandler}{event handler}}\\
\entry{like}{(\textit{пр.}) вподобайка}

\section{M}
\entry{manga}{(\textit{пр.}) японські мальописи}\\
\entry{menu}{\textbf{1.} (\textit{переносно}) спис, меню́; \textbf{2.} спис (-су) страв, стравоспис, ка́рта (ка́ртка) страв, меню́}\\
\entry{merge}{об’єднувати}\\
\entry{message}{повідомлення}\\
\entry{music video}{виднограй}\\
\entry{mutually}{навзає́м}

\section{N}
\entry{network}{сі́тка, сіть}\\
\subentry{social \textasciitilde}{соція́льна сі́тка, (\textit{пр.}) я́тір (\textit{р.} я́тера)}\\
\entry{node}{\textbf{1.} ву́зол, вузлови́й; \textbf{2.} верши́на (графу)}

\section{P}
\entry{package}{(у)пако́вання}\\
\entry{packing}{(у)пакува́ння}\\
\entry{path}{путь (\textit{ж. р.})}\\
\entry{phone}{\textit{див.} \hyperlink{telephone}{telephone}}\\
\entry{police}{полі́ція}\\
\subentry{\textasciitilde\ state}{поліці́йна дер\-жа́\-ва}\\
\entry{policeman}{поліція́н(т)}\\
\entry{postapocalypse}{постапо\-ка́\-ліпсис}\\
\entry{power bank}{наснажник}\\
\entry{problem}{за(в)да́ча}\\
\entry{project}{проє́кт}\\
\subentry{pet \textasciitilde}{ха́тній проє́кт}

\section{Q}
\entry{queue}{че́рга́}

\section{R}
\entry{real-time strategy}{страте́гія в реа́льному ча́сі}\\
\entry{real-time tactics}{тактика в реальному часі}\\
\entry{review}{о́гляд}\\
\entry{rose tree}{\textit{див.} \hyperlink{rosetree}{tree}}\\
\entry{RPG}{\textit{див.} \hyperlink{rpg}{role-playing game}}\\
\entry{russia}{моско́вщина (-ни), (\textit{зн.}) каца́пщина}\\
\hypertarget{re}{\entry{russian empire}{(\textit{зн.}) каца́пщина}}\\
\entry{russian federation}{\textit{див.} \hyperlink{re}{russian empire}}

\section{S}
\entry{screenshot}{(\textit{пр.}) зняток}\\
\entry{send a message}{посла́ти повідомлення}\\
\entry{setup}{підготування для роботи}\\
\entry{share}{поши́рити}\\
\entry{softcover}{м’яка́ окла́динка}\\
\entry{source code}{джерельний код}\\
\hypertarget{startup}{\entry{start-up}{(\textit{пр.}) розпочин}}\\
\entry{startup}{\textit{див.} \hyperlink{startup}{start-up}}\\
\entry{stop}{зу́пинка}\\
\entry{store}{\textbf{1.} крамни́ця; \textbf{2.} схо́вище}\\
\entry{string}{рядок}\\
\entry{subscribe}{підпи́суватися (підписа́тися)}\\
\entry{subscriber}{підписни́ця (підписни́к)}\\
\entry{subscription}{передпла́та}\\
\entry{support}{пі́дтримок}

\section{T}
\entry{tactical turn-based}{\textit{див.} \hyperlink{tbt}{turn-based tactics}}\\
\entry{task}{завда́ння́}\\
\hypertarget{telephone}{\entry{telephone}{ру́рка, (\textit{пр.}) у́жва, слухавка, телефон}}\\
\entry{topic}{те́ма}\\
\entry{tree}{дерево}\\
\hypertarget{rosetree}{\subentry{rose \textasciitilde}{трояндове дерево}}\\
\entry{tripod}{трині́жок (-жка), (\textit{штатив}) стоя́к}\\
\entry{turn-based strategy}{по\-кро́\-ко\-ва страте́гія}\\
\hypertarget{tbt}{\entry{turn-based tactics}{покро́кова тактика}}

\section{U}
\entry{upload}{заванта́ження}\\
\entry{urbanism}{містознавство}

\end{twocolumn}

\clearpage

\onecolumn
\section{Поясніння скорочень}
\subsection{Терміни і вислови}
\textit{див.} — дивись\\
\textit{док.} — доконаний вид\\
\textit{зн.} — зневажливо\\
\textit{пр.} — пропозиція\\
\textit{р.} — родовий відмінок

\thispagestyle{empty}

\end{document}
